\documentclass[8pt,a4paper,compress]{beamer}

\usepackage{/home/siyer/lib/slides}

\title{Collections and Functions}
\date{}

\begin{document}
\begin{frame}
\vfill
\titlepage
\end{frame}

\begin{frame}
\frametitle{Outline}
\tableofcontents
\end{frame}

\section{Lists}
\begin{frame}[fragile]
A \emph{collection} (aka \emph{container} or \emph{data structure}) is a way to organize data that we wish to process with a computer program

\bigskip

A \emph{list} (aka \emph{one-dimensional array} or \emph{array}) is a collection that stores a sequence of (references to) objects

\bigskip

The simplest way to create a list (an object of the built-in sequence type \lstinline{list}) in Python is to place comma-separated values between matching square brackets

\bigskip

For example, the following code creates a list \lstinline{suits} with four strings, and creates lists \lstinline{x} and \lstinline{y}, each with three floats

\begin{lstlisting}[language=Python]
suits = ['Clubs', 'Diamonds', 'Hearts', 'Spades']
x = [0.30, 0.60, 0.10]
y = [0.40, 0.10, 0.50]
\end{lstlisting}

\bigskip

After creating a list, we can refer to any individual object by specifying the list name followed by an integer index within square brackets

\bigskip

For example, if we have two lists of floats \lstinline{x} and \lstinline{y} whose length is given by a variable \lstinline{n}, we can calculate their dot product as follows

\begin{lstlisting}[language=Python]
total = 0.0
for i in range(n):
    total += x[i] * y[i]
\end{lstlisting} 
\end{frame}

\begin{frame}[fragile]
We refer to the first element of an $n$-element list \lstinline{a} as \lstinline{a[0]}, the second element as \lstinline{a[1]}, and so on; the last ($n$th) element is referred to as \lstinline{a[n - 1]}

\bigskip

We can access the length of a list \lstinline{a} using the built-in function \lstinline{len()}

\bigskip

We can use the \lstinline{+=} operator to append elements to a list; For example, the following code creates a list \lstinline{a} with $n$ floats, with each element initialized to \lstinline{0.0}

\begin{lstlisting}[language=Python]
a = []
for i in range(n):
    a += [0.0]
\end{lstlisting} 
\end{frame}

\begin{frame}[fragile]
Lists are \emph{mutable} objects because we can change their values; For example, the following code reverses the order of elements in a list \lstinline{a}

\begin{lstlisting}[language=Python]
n = len(a)
for i in range(n / 2):
    temp = a[i]
    a[i] = a[n - 1 - i]
    a[n - 1 - i] = temp
\end{lstlisting}

\bigskip

One of the most basic operations on a list is to \emph{iterate} over all its elements; For example, the following code computes the average of a list of floats

\begin{lstlisting}[language=Python]
total = 0.0
for i in range(len(a)):
    total += a[i]
average = total / len(a)
\end{lstlisting}

and the following code does the same without referring to the indices explicitly

\begin{lstlisting}[language=Python]
total = 0.0
for v in a:
    total += v
average = total / len(a)
\end{lstlisting}
\end{frame}

\begin{frame}[fragile]
Python has several built-in functions that take lists as arguments; For example, given a list \lstinline{a}
\begin{itemize}
\item \lstinline{len(a)} returns the number of elements in the list
\item \lstinline{sum(a)} returns the sum of the elements in the list
\item \lstinline{min(a)} returns the minimum element in the list
\item \lstinline{max(a)} returns the maximum element in the list
\item ...
\end{itemize}

\bigskip

We can write a list by passing it as an argument to \lstinline{stdio.write()} or \lstinline{stdio.writeln()}, or we can use a \lstinline{for} statement to write each element individually

\bigskip

\emph{Aliasing} refers to the situation where two variables refer to the same object; For example, after the assignment statements
\begin{lstlisting}[language=Python]
x = [.30, .60, .10]
y = x
x[1] = .99
\end{lstlisting}
the value of \lstinline{y[1]} is also \lstinline{.99}
\end{frame}

\begin{frame}[fragile]
One way to make a copy \lstinline{y} of a given list \lstinline{x} is to iterate though \lstinline{x} to build \lstinline{y}
\begin{lstlisting}[language=Python]
y = []
for v in x:
    y += [v]
\end{lstlisting}

\bigskip

Alternatively, the expression \lstinline{a[i:j]}, called \emph{slicing},  evaluates to a new list whose elements are \lstinline{a[i], ..., a[j - 1]}; The default value for \lstinline{i} is \lstinline{0} and the default value for \lstinline{j} is \lstinline{len(a)}, so \lstinline{y = x[:]} is equivalent to the code above

\bigskip

The \lstinline{stdarray} module defines functions for processing lists

\begin{center}
\begin{tabular}{cc}
function & description \\ \hline
\lstinline$stdarray.create1D(n, val)$ & list of length $n$, each element initialized to $val$ \\
\lstinline$stdarray.create2D(m, n, val)$ & $m$-by-$n$ list, each element initialized to $val$ \\
$\dots$ & $\dots$
\end{tabular} 
\end{center}

\bigskip

The \lstinline{sys.argv} object that we have been using to read command-line arguments is a list of string objects
\end{frame}

\begin{frame}[fragile]
\begin{framed}
\tiny Example: Code for representing and processing playing cards.
\end{framed}

Represent suits and ranks
\begin{lstlisting}[language=Python]
SUITS = ['Clubs', 'Diamonds', 'Hearts', 'Spades']
RANKS = ['2', '3', '4', '5', '6', '7', '8', '9', '10', 
         'Jack', 'Queen', 'King', 'Ace']
\end{lstlisting}

Write a random card name
\begin{lstlisting}[language=Python]
rank = random.randrange(0, len(RANKS))
suit = random.randrange(0, len(SUITS))
stdio.writeln(RANKS[rank] + ' of ' + SUITS[suit])
\end{lstlisting}

Create a deck
\begin{lstlisting}[language=Python]
deck = []
for rank in RANKS:
    for suit in SUITS:
        card = rank + ' of ' + suit
        deck += [card]
\end{lstlisting}

Shuffle the deck
\begin{lstlisting}[language=Python]
n = len(deck)
for i in range(n):
    r = random.randrange(i, n)
    temp = deck[r]
    deck[r] = deck[i]
    deck[i] = temp
\end{lstlisting}
\end{frame}

\begin{frame}[fragile]
\begin{framed}
\tiny couponcollector.py: Accept integer $n$ as a command-line argument. Write to standard output the number of coupons you collect before obtaining one of each of $n$ types.
\end{framed}

\begin{lstlisting}[language=Python]
import random
import stdarray
import stdio
import sys

n = int(sys.argv[1])
count = 0
collectedCount = 0
isCollected = stdarray.create1D(n, False)
while collectedCount < n:
    value = random.randrange(0, n)
    count += 1
    if not isCollected[value]:
        collectedCount += 1
        isCollected[value] = True
stdio.writeln(count)
\end{lstlisting}

\begin{lstlisting}[language={}]
$ python couponcollector.py 1000000
13988284
\end{lstlisting}
\end{frame}

\begin{frame}[fragile]
In many applications, a convenient way to store information is to use a table of numbers organized in a rectangular table and refer to \emph{rows} and \emph{columns} in the table

\bigskip

The simplest way to create a two-dimensional list in Python is to place comma-separated one-dimensional lists between matching square brackets 

\bigskip

For example, this matrix of integers having two rows and three columns 
\[
\begin{bmatrix}
18 & 19 & 20 \\ 
21 & 22 & 23
\end{bmatrix}
\]
can be represented in Python using this list of lists
\begin{lstlisting}[language=Python]
a = [[18, 19, 20], [21, 22, 23]]
\end{lstlisting}

\bigskip

Python represents an $m$-by-$n$ list as a list that contains $m$ objects, each of which is a list that contains $n$ objects

\bigskip

For example, the following code creates an $m$-by-$n$ list \lstinline{a} of floats, with all elements initialized to \lstinline{0.0}
\begin{lstlisting}[language=Python]
a = []
for i in range(m):
    row = [0.0] * n
    a += [row]
\end{lstlisting}
\end{frame}

\begin{frame}[fragile]
When \lstinline{a} is a two-dimensional list, the syntax \lstinline{a[i]} refers to its \lstinline{i}th row, which is a one-dimensional list; The syntax \lstinline{a[i][j]} refers to the object at row \lstinline{i} and column \lstinline{j}

\bigskip

For example, the following code adds two $n$-by-$n$ matrices \lstinline{a} and \lstinline{b}

\begin{lstlisting}[language=Python]
c = stdarray.create2D(n, n, 0.0)
for i in range(n):
    for j in range(n):
        c[i][j] = a[i][j] + b[i][j]
\end{lstlisting}
\end{frame}

\begin{frame}[fragile]
\begin{framed}
\tiny selfavoid.py: Accept integers $n$ and $trials$ as command-line arguments. Do $trials$ random self-avoiding walks in an $n$-by-$n$ lattice. Write to standard output the percentage of dead ends encountered.
\end{framed}

\begin{lstlisting}[language=Python]
import random
import stdarray
import stdio
import sys

n      = int(sys.argv[1])
trials = int(sys.argv[2])
deadEnds = 0
for t in range(trials):
    a = stdarray.create2D(n, n, False)
    x = n / 2
    y = n / 2
    while (x > 0) and (x < n - 1) and (y > 0) and (y < n - 1):
        a[x][y] = True
        if a[x - 1][y] and a[x + 1][y] and a[x][y - 1] and a[x][y + 1]:
            deadEnds += 1
            break
        r = random.randrange(1, 5)
        if   (r == 1) and (not a[x + 1][y]):
            x += 1
        elif (r == 2) and (not a[x - 1][y]):
            x -= 1
        elif (r == 3) and (not a[x][y + 1]):
            y += 1
        elif (r == 4) and (not a[x][y - 1]):
            y -= 1
stdio.writeln(str(100 * deadEnds / trials) + '% dead ends')
\end{lstlisting}
\end{frame}

\begin{frame}[fragile]
\begin{lstlisting}[language={}]
$ python selfavoid.py 5 100
0% dead ends
$ python selfavoid.py 20 100
31% dead ends
$ python selfavoid.py 40 100
77% dead ends
$ python selfavoid.py 80 100
98% dead ends
$ python selfavoid.py 5 1000
0% dead ends
$ python selfavoid.py 20 1000
30% dead ends
$ python selfavoid.py 40 1000
75% dead ends
$ python selfavoid.py 80 1000
98% dead ends
\end{lstlisting}
\end{frame}

\begin{frame}[fragile]
A list with rows of nonuniform length is known as a \emph{ragged list}

\bigskip

For example, the following code writes the contents of a ragged list

\begin{lstlisting}[language=Python]
for i in range(len(a)):
    for j in range(len(a[i])):
        stdio.write(a[i][j])
        stdio.write(' ')
    stdio.writeln()
\end{lstlisting}

\bigskip

The same notation extends to allow us to compose code using lists that have any number of dimensions; For example, using lists of lists of lists, we can create a three-dimensional list \lstinline{a}, and then refer to an individual element of \lstinline{a} as \lstinline{a[i][j][k]}
\end{frame}

\section{Tuples}
\begin{frame}[fragile]
A \emph{tuple} (an object of the built-in sequence type \lstinline{tuple}) consists of a number of values separated by commas
\begin{lstlisting}[language={}]
>>> t = 12345, 54321, 'hello!'
>>> t[0]
12345
>>> t
(12345, 54321, 'hello!')
\end{lstlisting}

\bigskip

Tuples may be nested
\begin{lstlisting}[language={}]
>>> u = t, (1, 2, 3, 4, 5)
>>> u
((12345, 54321, 'hello!'), (1, 2, 3, 4, 5))
\end{lstlisting}

\bigskip

Tuples are immutable, but they can contain mutable objects
\begin{lstlisting}[language={}]
>>> v = ([1, 2, 3], [3, 2, 1])
>>> v
([1, 2, 3], [3, 2, 1])
\end{lstlisting}

\bigskip

Empty and singleton sequences
\begin{lstlisting}[language={}]
>>> empty = ()
>>> singleton = 'hello', 
>>> len(empty)
0
>>> len(singleton)
1
\end{lstlisting}

\bigskip

Sequence unpacking
\begin{lstlisting}[language={}]
>>> x, y, z = t
\end{lstlisting}
\end{frame}

\section{Sets}
\begin{frame}[fragile]
A \emph{set} (an object of the built-in sequence type \lstinline{set}) is an unordered collection with no duplicate elements
\begin{lstlisting}[language={}]
>>> basket = ['apple', 'orange', 'apple', 'pear', 'orange', 'banana']
>>> fruit = set(basket)
>>> fruit
set(['orange', 'pear', 'apple', 'banana'])
\end{lstlisting}

\bigskip

Fast membership testing
\begin{lstlisting}[language={}]
>>> 'orange' in fruit
True
>>> 'crabgrass' in fruit
False
\end{lstlisting}

\bigskip

Set operations
\begin{lstlisting}[language={}]
>>> a = set('abracadabra')
>>> b = set('alacazam')
>>> a                                  # unique letters in a
set(['a', 'r', 'b', 'c', 'd'])
>>> a - b                              # letters in a but not in b
set(['r', 'd', 'b'])
>>> a | b                              # letters in either a or b
set(['a', 'c', 'r', 'd', 'b', 'm', 'z', 'l'])
>>> a & b                              # letters in both a and b
set(['a', 'c'])
>>> a ^ b                              # letters in a or b but not both
set(['r', 'd', 'b', 'm', 'z', 'l'])
\end{lstlisting}
\end{frame}

\section{Dictionaries}
\begin{frame}[fragile]
A \emph{dictionary} (an object of the built-in mapping type \lstinline{dict}) is an unordered set of \emph{key}-\emph{value} pairs, with the requirement that the keys are unique
\begin{lstlisting}[language={}]
>>> tel = {'jack': 4098, 'sape': 4139}
>>> tel['guido'] = 4127
>>> tel
{'sape': 4139, 'guido': 4127, 'jack': 4098}
>>> tel['jack']
4098
>>> tel['irv'] = 4127
>>> tel
{'guido': 4127, 'irv': 4127, 'jack': 4098, 'sape': 4139}
>>> 'guido' in tel
True
\end{lstlisting}

\bigskip

Dictionaries can be built directly from sequences of key-value pairs
\begin{lstlisting}[language={}]
>>> dict([('sape', 4139), ('guido', 4127), ('jack', 4098)])
{'sape': 4139, 'jack': 4098, 'guido': 4127}
\end{lstlisting}

\end{frame}

\section{Comprehensions}
\begin{frame}[fragile]
\emph{Comprehensions} provide a concise way to create collections

\bigskip

List comprehensions
\begin{lstlisting}[language={}]
>>> squares = [x ** 2 for x in range(10)]
>>> squares
[0, 1, 4, 9, 16, 25, 36, 49, 64, 81]
>>> [(x, y) for x in [1, 2, 3] for y in [3, 1, 4] if x != y]
[(1, 3), (1, 4), (2, 3), (2, 1), (2, 4), (3, 1), (3, 4)]
\end{lstlisting}

\bigskip

Nested list comprehensions
\begin{lstlisting}[language={}]
>>> matrix = [
...     [1, 2, 3, 4],
...     [5, 6, 7, 8],
...     [9, 10, 11, 12],
... ]
>>> [[row[i] for row in matrix] for i in range(4)]
[[1, 5, 9], [2, 6, 10], [3, 7, 11], [4, 8, 12]]
\end{lstlisting}

\bigskip

Set comprehensions
\begin{lstlisting}[language={}]
>>> a = {x for x in 'abracadabra' if x not in 'abc'}
>>> a
set(['r', 'd'])
\end{lstlisting}

\bigskip

Dictionary comprehensions
\begin{lstlisting}[language={}]
>>> {x: x ** 2 for x in (2, 4, 6)}
{2: 4, 4: 16, 6: 36}
\end{lstlisting}
\end{frame}

\section{Looping Techniques}
\begin{frame}[fragile]
When looping through a sequence, the position index and corresponding value can be retrieved at the same time using the \lstinline{enumerate()} function
\begin{lstlisting}[language={}]
>>> for i, v in enumerate(['tic', 'tac', 'toe']):
...     stdio.writeln(str(i) + ' ' + v)
...
0 tic
1 tac
2 toe
\end{lstlisting}

\bigskip

To loop over two or more sequences at the same time, the entries can be paired with the \lstinline{zip()} function
\begin{lstlisting}[language={}]
>>> questions = ['name', 'quest', 'favorite color']
>>> answers = ['lancelot', 'the holy grail', 'blue']
>>> for q, a in zip(questions, answers):
...     stdio.writeln('What is your ' + q + '? It is ' + a)
...
What is your name? It is lancelot.
What is your quest? It is the holy grail.
What is your favorite color? It is blue.
\end{lstlisting}
\end{frame}

\begin{frame}[fragile]
To loop over a sequence in reverse, first specify the sequence in a forward direction and then call the \lstinline{reversed()} function
\begin{lstlisting}[language={}]
>>> for i in reversed(range(1, 10, 2)):
...     stdio.writeln(i)
...
9
7
5
3
1
\end{lstlisting}

\bigskip

To loop over a sequence in sorted order, use the \lstinline{sorted()} function which returns a new sorted list while leaving the source unaltered
\begin{lstlisting}[language={}]
>>> basket = ['apple', 'orange', 'apple', 'pear', 'orange', 'banana']
>>> for f in sorted(set(basket)):
...     stdio.writeln(f)
...
apple
banana
orange
pear
\end{lstlisting}
\end{frame}

\section{Using and Defining Functions}
\begin{frame}[fragile]
Functions allow us to
\begin{itemize}
\item Transfer control back and forth between different pieces of code 
\item Clearly separate tasks within a program
\item Reuse code
\end{itemize}

\bigskip

Function definition in Python
\begin{lstlisting}[language=Python]
def <name>(<parameter name>, <parameter name>, ...):
    <statement>
    <statement>
    ...
\end{lstlisting}

\begin{itemize}
\item The first line, known as the \emph{signature}, consists of the \lstinline{def} keyword, the function name, a sequence of zero or more parameter variables separated by commas and enclosed in parentheses, and a colon

\item The indented statements following the signature constitute the \emph{function body}

\item The function body can contain a \emph{return statement}, which transfers control to the point where the function was called and returns the result of the computation (aka \emph{the return value})
\end{itemize}
\end{frame}

\begin{frame}[fragile]
For example, the following function tests if the argument $N$ is prime

\begin{lstlisting}[language=Python]
def is_prime(N):
    if N < 2: 
        return False
    i = 2
    while i <= N / i:
        if N % i == 0:
            return False
        i += 1
    return True
\end{lstlisting}

\bigskip

A \emph{function call} is its name followed by expressions that specify argument values in parentheses, separated by commas; For example, \lstinline{is_prime(31)}

\bigskip

When the function call is part of an expression, as in \lstinline{x = math.sqrt(3)}, the function computes a value that is used in place of the call in the expression

\bigskip

Otherwise, the function call is a statement that generally causes side effects, as in \lstinline{stdio.writeln('Hello, World')}

\bigskip

When a module is executed directly by the \lstinline{python} command (and not via an \lstinline{import} statement), the module's \lstinline{__name__} is set equal to \lstinline{'__main__'}
\end{frame}

\begin{frame}[fragile]
\begin{framed}
\tiny harmonicf.py:  Write to standard output the harmonic numbers specified as command-line arguments.
\end{framed}

\begin{lstlisting}[language=Python]
import stdio
import sys

def harmonic(n):
    total = 0.0
    for i in range(1, n + 1):
        total += 1.0 / float(i)
    return total

def main():
    for j in range(1, len(sys.argv)):
        arg = int(sys.argv[j])
        value = harmonic(arg)
        stdio.writeln(value)

if __name__ == '__main__':
    main()
\end{lstlisting}

\begin{lstlisting}[language={}]
$ python harmonicf.py 1 2 4
1.0
1.5
2.08333333333
\end{lstlisting}
\end{frame}

\begin{frame}[fragile]
A Python function can have more than one parameter variable, so it can be called with more than one argument
\begin{lstlisting}[language=Python]
def hypot(a, b):
    return math.sqrt(a * a + b * b)
\end{lstlisting}

\bigskip

We can define as many functions as we want in a \lstinline{.py} file
\begin{lstlisting}[language=Python]
def square(x):
    return x * x

def hypot(a, b):
    return math.sqrt(square(a) + square(b))
\end{lstlisting}

\bigskip

We can put \lstinline{return} statements in a function wherever we need them, as in the \lstinline{is_prime()} function

\bigskip

A function provides only one return value to the caller; More precisely, it returns a reference to one object

\bigskip

The scope of a function's local and parameter variables is limited to that function

\bigskip

The scope of a variable defined in global code --- known as a \emph{global variable} --- is limited to the \lstinline{.py} file containing that variable
\end{frame}

\begin{frame}[fragile]
A function may designate an argument to be \emph{optional} by specifying a \emph{default value} for that argument

\bigskip

For example, the following function computes the \emph{$n$th generalized harmonic number of order $r$}, $H_{n,r}=1+1/2^r+1/3^r+\cdots+1/n^r$

\begin{lstlisting}[language=Python]
def harmonic(n, r = 1):
    total = 0.0
    for i in range(1, n + 1):
        total += 1.0 / (i ** r)
    return total
\end{lstlisting}
With this definition, \lstinline{harmonic(2, 2)} returns \lstinline{1.25}, while both \lstinline{harmonic(2, 1)} and \lstinline{harmonic(2)} return \lstinline{1.5}

\bigskip

\emph{Named arguments} allow us to specify arguments to a function in any order; For example, the call \lstinline{harmonic(r = 2, n = 3)} is the same as the call \lstinline{harmonic(3, 2)}
\end{frame}

\begin{frame}[fragile]
\begin{framed}
\tiny coupon.py: Accept integer $n$ as a command-line argument. Write to standard output the number of coupons collected before obtaining one of each of $n$ types.
\end{framed}

\begin{lstlisting}[language=Python]
import random
import stdarray
import stdio
import sys

def getCoupon(n):
    return random.randrange(0, n)

def collect(n):
    found = stdarray.create1D(n, False)
    couponCount = 0
    distinctCouponCount = 0
    while distinctCouponCount < n:
        coupon = getCoupon(n)
        couponCount += 1
        if not found[coupon]:
            distinctCouponCount += 1
            found[coupon] = True
    return couponCount

def main():
    n = int(sys.argv[1])
    couponCount = collect(n)
    stdio.writeln(couponCount)

if __name__ == '__main__':
    main()
\end{lstlisting}

\begin{lstlisting}[language={}]
$ python coupon.py 1000000
15490879
\end{lstlisting}
\end{frame}

\begin{frame}[fragile]
We can use parameter variables anywhere in the body of the function in the same way as we use local variables

\bigskip

Python initializes the parameter variable with the corresponding argument provided by the calling code --- an approach known as \emph{call by object reference} (aka \emph{call by value})

\bigskip

As a consequence, if a parameter variable refers to a mutable object and we change that object's value within a function, then this also changes the object's value in the calling code

\bigskip

When a function takes a list as an argument, it implements a function that operates on an arbitrary number of objects
\begin{lstlisting}[language=Python]
def mean(a):
    total = 0.0
    for v in a:
        total += v
    return total / len(a)
\end{lstlisting}
\end{frame}

\begin{frame}[fragile]
Since lists are \emph{mutable}, it is often the case that the purpose of a function that takes a list as argument is to produce a side effect
\begin{lstlisting}[language=Python]
def exchange(a, i, j):
    temp = a[i]
    a[i] = a[j]
    a[j] = temp
\end{lstlisting}

\bigskip

There are situations where it is useful for a function to return a list
\begin{lstlisting}[language=Python]
def randomarray(n):
    a = stdarray.create1D(n)
    for i in range(n):
        a[i] = random.random()
    return a
\end{lstlisting}
\end{frame}

\section{Filter, Lambda, Map, and Reduce Functions}
\begin{frame}[fragile]
In Python, functions are \emph{first-class citizens}, meaning they are actually data just like numbers, lists, and strings

\bigskip

As a result, we can write functions that take \emph{other functions} as arguments and return \emph{other functions} as results

\bigskip

The built-in function \lstinline{filter(f, seq)} returns those items of the sequence \lstinline{seq} for which \lstinline{f(item)} is \lstinline{True}
\begin{lstlisting}[language={}]
>>> primes = filter(is_prime, range(11))
>>> primes
[2, 3, 5, 7]
\end{lstlisting}

\bigskip

A \emph{lambda function} is a ``disposable'' function that we can define just when we need it and then immediately throw it away after we are done using it
\begin{lstlisting}[language={}]
>>> odds = filter(lambda x : x % 2 != 0, range(11))
>>> odds
[1, 3, 5, 7, 9]
\end{lstlisting}
\end{frame}

\begin{frame}[fragile]
The built-in function \lstinline{map(f, seq)} returns a list of the results of applying the function \lstinline{f} to the items of the sequence \lstinline{seq}
\begin{lstlisting}[language={}]
squares = map(lambda x : x ** 2, range(11))
>>> squares
[0, 1, 4, 9, 16, 25, 36, 49, 64, 81, 100]
\end{lstlisting}

\bigskip

The built-in function \lstinline{reduce(f, seq)} applies a function \lstinline{f} of two arguments cumulatively to the items of a sequence \lstinline{seq}, from left to right, so as to reduce the sequence to a single value
\begin{lstlisting}[language={}]
>>> total = reduce(lambda x, y: x + y, range(11))
>>> total
55
\end{lstlisting}
\end{frame}

\section{Exercises}
\begin{frame}[fragile]
\begin{enumerate}
\item Write a program \lstinline{stats.py} that writes the mean $\mu$, variance $Var$, and standard deviation $\sigma$ of the floats $x_1, 
x_2, \dots, x_n$ supplied as command-line arguments, computed using the formulae:

\[
\mu = \frac{1}{n}\sum\limits_i^n x_i, 
Var = \frac{1}{n}\sum\limits_i^n (x_i - \mu_i)^2, \text{ and }
\sigma = \sqrt{Var}.
\]

\begin{lstlisting}[language={}]
$ python stats.py 77 41 3 76 19 48 92 10 26 63
45.5
856.65
29.26858384
\end{lstlisting}

\item Suppose that people enter an empty room until a pair of people share a birthday. On average, how many people will have to enter 
before there is a match? Write a program \lstinline{birthday.py} that takes an integer $trials$ from the command line and runs $trials$ experiments to estimate this qua
ntity, where each experiment involves sampling individuals until a pair of them share a birthday. Assume birthdays to be uniform random integers from the interval $[0, 
365)$. 

\begin{lstlisting}[language={}]
$ python birthday.py 1000
24
\end{lstlisting}
\end{enumerate}
\end{frame}

\begin{frame}[fragile]
\begin{enumerate}\setcounter{enumi}{2}
\item Pascal's triangle $\mathcal{P}_n$ is a triangular array with $n+1$ rows, each listing the coefficients of the binomial expansio
n $(x+y)^i$, where $0 \leq i \leq n$. For example, $\mathcal{P}_4$ is the triangular array:
\[
\begin{matrix}
1 & & & \\
1 & 1 & & \\
1 & 2 & 1 & \\
1 & 3 & 3 & 1 \\
1 & 4 & 6 & 4 & 1
\end{matrix}
\]
The term $\mathcal{P}_n(i, j)$ is calculated as $\mathcal{P}_n(i-1, j-1)+\mathcal{P}_n(i-1, j),$ where $0 \leq i \leq n$ and $1 \leq j < i$, with $\mathcal{P}_n(i, 0) =
 \mathcal{P}_n(i, i) = 1$ for all $i$. Write a program \lstinline{pascal.py} that takes an integer $n$ as command-line argument and writes $\mathcal{P}_n$.

\begin{lstlisting}[language={}]
$ python pascal.py 10
1
1 1
1 2 1
1 3 3 1
1 4 6 4 1
1 5 10 10 5 1
1 6 15 20 15 6 1
1 7 21 35 35 21 7 1
1 8 28 56 70 56 28 8 1
1 9 36 84 126 126 84 36 9 1
1 10 45 120 210 252 210 120 45 10 1
\end{lstlisting}
\end{enumerate}
\end{frame}

\begin{frame}[fragile]
\begin{enumerate}\setcounter{enumi}{3}
\item Implement the function \lstinline{distance()} in \lstinline{distance.py} that returns the Euclidean distance between the vecto
rs $x$ and $y$ represented as one-dimensional lists of floats. The Euclidean distance is calculated as the square root of the sums of the squares of the differences bet
ween the corresponding entries. You may assume that $x$ and $y$ have the same length.

\begin{lstlisting}[language={}]
$ python distance.py 5
-9 1 10 -1 1
-5 9 6 7 4
<ctrl-d>
13.0
\end{lstlisting}

\item Implement the function \lstinline{is_palindrome()} in \lstinline{palindrome.py} that returns \lstinline{True} if the argument $s$ is a
 palindrome (ie, reads the same forwards and backwards), and \lstinline{False} otherwise. You may assume that $s$ is all lower case and doesn't any whitespace character
s.

\begin{lstlisting}[language={}]
$ python palindrome.py bolton
False
$ python palindrome.py amanaplanacanalpanama
True
\end{lstlisting}
\end{enumerate}
\end{frame}

\begin{frame}[fragile]
\begin{enumerate}\setcounter{enumi}{5}
\item Implement the function \lstinline{transpose()} in \lstinline{transpose.py} that creates and returns a new matrix that is the transpose 
of the matrix represented by the argument $a$. Note that $a$ need not have the same number rows and columns. Recall that the transpose of an $m$-by-$n$ matrix $A$ is an
 $n$-by-$m$ matrix $B$ such that $B_{ij}=A_{ji}$, where $1 \leq i \leq n$ and $1 \leq j \leq m$.

\begin{lstlisting}[language={}]
$ python transpose.py 
2 3 
1 2 3
4 5 6
1.0 4.0
2.0 5.0
3.0 6.0
\end{lstlisting}
\end{enumerate}
\end{frame}
\end{document}
